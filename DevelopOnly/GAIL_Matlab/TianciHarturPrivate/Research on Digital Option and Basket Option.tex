\documentclass[]{elsarticle}
\setlength{\marginparwidth}{0.5in}
\usepackage{amsmath,amssymb,amsthm,mathtools,tikz,graphicx}


\DeclareMathOperator{\lin}{lin}
\DeclareMathOperator{\up}{up}
\DeclareMathOperator{\lo}{lo}
\DeclareMathOperator{\fix}{non}
\DeclareMathOperator{\err}{err}

\newtheorem{theorem}{Theorem}
\newtheorem{prop}[theorem]{Proposition}
\newtheorem{lem}{Lemma}
\theoremstyle{definition}
\newtheorem{algo}{Algorithm}
\newtheorem{condit}{Condition}
%\newtheorem{assump}{Assumption}
\theoremstyle{remark}
\newtheorem{rem}{Remark}


\begin{document}

\begin{frontmatter}
\title{Research on Digital Option and Basket Option}
\author{Tianci Zhu}
\author{Hartur Santi}
\address{Department of Applied Mathematics, Illinois Institute of Technology, Illinois, USA}

\begin{abstract}
Option is a common financial derivative, which has many different types. The digital option and basket option in this article are both traded as European style, meaning that contracts are only exercised on expiry.

Several approaches are proceeded in the past to price digital options and basket options, such as the Monte Carlo Method and Black-Scholes Model. This article applies the Monte Carlo Method to both of the options and uses Black-Scholes Model to test the accuracy of Monte Carlo Method.
\end{abstract}

\begin{keyword}
Digital option, Basket Option, Monte Carlo Method, Quasi Monte Carlo Method, Black-Scholes Model
\end{keyword}
\end{frontmatter}

\section{Introduction}
\subsection{Option}

\subsection{Monte Carlo Method}

\subsection{Quasi-Monte Carlo Method}

\subsection{Digital Option}

\subsection{Basket Option}
Basket option is a type of option  which contains a number of underlying assets. Basket option payoff is calculated as follows:

For a call option:
\[max(\sum_{i=1}^n {a_i*S_T^{(i)}} - K, 0)\]

For a put option:
 \[max(K - \sum_{i=1}^n {a_i*S_T^{(i)}}, 0)\]
 
 where:
 
 \(n\) - number of assets
 
 \(K\) - strike price\\
 
 
As there is no formula to calculate the exact price of basket options, this article will only talk about the Monte Carlo Method and Quasi-Monte Carlo Method on basket option.

\section{Pricing Options}
To price an option, four regular steps are needed:
\begin{itemize}
\item Add needed parameters 
\item Set values to parameters
\item Add payoffs corresponding to the payoff formulas mentioned in the introduction part
\item Add exact price calculation if exact formula exists
\end{itemize}
\subsection{Digital Option}

\subsubsection{Black-Scholes Formula }

\subsubsection{Monte Carlo Method}

\subsection{Basket Option}
To calculate the approximate option price of basket option, some properties are added to MATLAB files:
\begin{description}
\item[nAsset]--- Number of assets in the option
\item[corrMat]--- Correlation Matrix of assets in the option
\item[basketWeight]--- Asset weights in the option 
\end{description}

As shown in the introduction part, the payoff of a basket option is calculated using the weighted stock prices of assets in the option, then the approximate price of the basket option is generated using Monte Carlo Method.

\section{Testing}
\subsection{Digital Option}

\subsection{Basket Option}
When the elements of correlation matrix are all ones, the assets in the basket option can be seen as the same asset, so the option price can be calculated as a European option, which is used as the testing method for basket options.

To get the size of samples larger, nested for loops are used in the test file. Both Monte Carlo Method and Quasi-Monte Carlo Method are tested in the file. The test takes about 50 seconds to run and it tests 162 different options for Monte Carlo Method and Quasi-Monte Carlo Method respectively.

\section*{Appendix}
\subsection*{optPayoff}
\subsection*{Test}
\subsubsection*{Digital Option}
\subsubsection*{Basket Option}
classdef BasketOptionTest < matlab.unittest.TestCase\\
      % BasketOptionTest tests the basket option price generated using\\
      % Monte Carlo Method and Quasi Monte Carlo Method.\\
    
    
      methods (Test)\\
          % Test Monte Carlo Method\\
          function testIIDSolution(testCase)\\
              % Assign parameter values for european option\\
              euroinp.payoffParam.optType = {'euro'};\\
              euroinp.timeDim.timeVector = 1;\\
              euroinp.priceParam.absTol = 0;\\
              euroinp.priceParam.relTol = 0.01;\\
              euroinp.timeDim.timeVector = 1;\\
              euroinitPrice=[10 11 12];\\
              eurovolatility=[0.5 0.6 0.3];\\
              strike=[9 10 11];\\
              callPut=[{'call'},{'put'}];\\
              nOption=length(euroinitPrice);\\
              q=0;\\
              for i=1:nOption\\
                  for j=1:nOption\\
                      for m=1:nOption\\
                          for n=1:2\\
                              for v=1:nOption\\
                              euroinp.payoffParam.putCallType = callPut(n);\\
                              euroinp.payoffParam.strike = strike(m);\\
                              euroinp.assetParam.volatility = eurovolatility(j);\\
                              euroinp.assetParam.initPrice = euroinitPrice(i);\\
                              eurooption = optPayoff(euroinp);\\
                              q = q+1;\\
                              exactPrice(q)=eurooption.exactPrice;
                              end\\
                          end\\
                      end\\
                  end\\
              end\\
              % Set parameter value for basket option
              inp.payoffParam.optType = {'basket'};\\
              inp.assetParam.nAsset = 2;\\
              inp.assetParam.corrMat = [1 1;1 1];\\
              inp.priceParam.absTol = 0;  \\
              inp.priceParam.relTol = 0.01;\\
              inp.timeDim.timeVector = 1;\\
              basketinitPrice = repmat(euroinitPrice',1,inp.assetParam.nAsset);\\
              basketvolatility = repmat(eurovolatility',1,inp.assetParam.nAsset);\\
              basketWeight = [0.3 0.7; 0.5 0.5; 0.2 0.8];\\
              p=0;\\
              for a=1:nOption\\
                  for b=1:nOption\\
                      for c=1:nOption\\
                          for d=1:2\\
                              for e=1:nOption\\
                                 inp.payoffParam.basketWeight = basketWeight(e,:);\\
                                 inp.payoffParam.putCallType = callPut(d);\\
                                 inp.payoffParam.strike = strike(c);\\
                                 inp.assetParam.volatility = basketvolatility(b,:);\\
                                 inp.assetParam.initPrice = basketinitPrice(a,:);\\
                                 p=p+1;\\
                                 BasketOption = optPrice(inp);\\
                                 % calculate basket option\\
                                 appPrice(p) = genOptPrice(BasketOption);\\
                              end\\
                          end\\
                      end\\
                  end\\
              end\\
              
              % Test the relative error
              testCase.verifyLessThan(abs(appPrice-exactPrice)./exactPrice,BasketOption.priceParam.relTol);\\
          end\\
          
            % Test Quasi-Monte Carlo Method
          function testSobolSolution(testCase)\\
              % Assign parameter values for european option
              euroinp.payoffParam.optType = {'euro'};\\
              euroinp.timeDim.timeVector = 1;\\
              euroinp.priceParam.absTol = 0;\\
              euroinp.priceParam.relTol = 0.01;\\
              euroinp.timeDim.timeVector = 1;\\
              euroinitPrice=[10 11 12];\\
              eurovolatility=[0.5 0.6 0.3];\\
              strike=[9 10 11];\\
              callPut=[{'call'},{'put'}];\\
              nOption=length(euroinitPrice);\\
              q=0;\\
              for i=1:nOption\\
                  for j=1:nOption\\
                      for m=1:nOption\\
                          for n=1:2\\
                              for v=1:nOption\\
                              euroinp.payoffParam.putCallType = callPut(n);\\
                              euroinp.payoffParam.strike = strike(m);\\
                              euroinp.assetParam.volatility = eurovolatility(j);\\
                              euroinp.assetParam.initPrice = euroinitPrice(i);\\
                              eurooption = optPayoff(euroinp);\\
                              q = q+1;\\
                              exactPrice(q)=eurooption.exactPrice;
                              end\\
                          end\\
                      end\\
                  end\\
              end\\
              % Set parameter value for basket option
              inp.payoffParam.optType = {'basket'};\\
              inp.assetParam.nAsset = 2;\\
              inp.assetParam.corrMat = [1 1;1 1];\\
              inp.priceParam.cubMethod='Sobol';\\
              inp.priceParam.absTol = 0;   \\
              inp.priceParam.relTol = 0.01;\\
              inp.timeDim.timeVector = 1;\\
              basketinitPrice = repmat(euroinitPrice',1,inp.assetParam.nAsset);\\
              basketvolatility = repmat(eurovolatility',1,inp.assetParam.nAsset);\\
              basketWeight = [0.3 0.7; 0.5 0.5; 0.2 0.8];\\
              p=0;\\
              for a=1:nOption\\
                  for b=1:nOption\\
                      for c=1:nOption\\
                          for d=1:2\\
                              for e=1:nOption\\
                                 inp.payoffParam.basketWeight = basketWeight(e,:);\\
                                 inp.payoffParam.putCallType = callPut(d);\\
                                 inp.payoffParam.strike = strike(c);\\
                                 inp.assetParam.volatility = basketvolatility(b,:);\\
                                 inp.assetParam.initPrice = basketinitPrice(a,:);\\
                                 p=p+1;\\
                                 BasketOption = optPrice(inp);\\
                                 % calculate basket option
                                 appPrice(p) = genOptPrice(BasketOption);\\
                              end\\
                          end\\
                      end\\
                  end\\
              end\\
              % Test the relative error
              testCase.verifyLessThan(abs(appPrice-exactPrice)./exactPrice,BasketOption.priceParam.relTol);\\
          end\\
     end\\
  end \\

\section*{Reference}
"Binary Option." Wikipedia. Accessed June 5, 2015.

"What Are Binary Options?" Binary Options. Accessed June 16, 2015.

"Options Basics: What Are Options? | Investopedia." Investopedia. December 2, 2003. Accessed June 22, 2015.

"Online Option Pricer." Options Pricer. Web. 22 June 2015.

"Basket Option Definition | Investopedia." Investopedia. 19 Nov. 2003. Web. 22 June 2015.
\end{document}

