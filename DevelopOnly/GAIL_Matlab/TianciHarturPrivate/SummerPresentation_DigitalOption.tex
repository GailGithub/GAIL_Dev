\documentclass[10pt,Compress]{beamer}
\usetheme{Warsaw}
\usecolortheme{dolphin}
\title[Options]{Summer Research Presentation - Digital Options}
\author{Tianci Zhu \and Hartur Santi}
\institute[IIT]{Illinois Institute of Technology}
\begin{document}
\titlepage


\begin{frame}
\frametitle{Overview}
\tableofcontents
\end{frame}

\section{Matlab Class}
\begin{frame}
\frametitle{Matlab class}
\begin{itemize}
\item Class definition: Description of what is common to every instance of a class.
\item Properties: Data storage for class instances
\item Methods: Special functions that implement operations that are usually performed only on instances of the class
\end{itemize}
\end{frame}

\begin{frame}
\frametitle{Examples form optPayoff.m}
\begin{itemize}
\item classdef optPayoff $<$ assetPath
\item properties (SetAccess=public) \\
\hspace{0.5 cm} payoffParam = struct('optType', {{'euro'}}, ...\\
 \hspace{1.2cm}  'putCallType', {{'call'}}, ...\\ 
 \hspace{1.2cm}       'strike', 10, ... \\
 \hspace{1.2cm}        'barrier', 12, ... \\
 \hspace{1.2cm}      'digitalRate', 0.5, ...\\
 \hspace{1.2cm}     'cashAssetType',{{'cash'}}); 
 \item  methods (Access = protected)\\
 
      \hspace{0.5 cm}function propList = getPropertyList(obj)\\
     \hspace{0.5 cm} ......\\
         \hspace{0.5 cm}end\\
   end
 \end{itemize}
\end{frame}

\section{Digital Option}
\begin{frame}
\frametitle{Definition}
Digital Option is an option whose payout is fixed after the underlying stock exceeds the predetermined threshold or strike price. 
\end{frame}

\begin{frame}
\frametitle{Process}
\begin{itemize}
\item Set default properties
\item Set the properties of the payoff object
\item Generate payoffs of options
\item Calculate exact price
\item Add new properties to the list 
\item Generate option prices using Monte Carlo Method and compare if the error is within expected tolerance
\end{itemize}
\end{frame}

\begin{frame}
\frametitle{Formula}
In the Black–Scholes model, the price of the option can be found by the formulas below.
$$\Phi(x)=1/\sqrt{(2\pi)}\int_{-\infty}^{x}e^{-\frac{1}{2}z^2}dz$$
and,
$$d_1=\frac{ln(\frac{S}{K})+(r-q+\sigma^2/2)T}{\sigma\sqrt{T}}, d_2=d_1-\sigma\sqrt{T}$$
Where:\\
$S$ - initial stock price\\
$K$ - strike price\\
$T$ - time to maturity\\
$q$ - dividend rate\\
$r$ - risk-free interest rate\\
$\sigma$ - volatility\\
$\Phi$ - cumulative distribution function of the normal distribution
\end{frame}

\begin{frame}
Cash-or-nothing call\\
$$C=e^{-rT}\Phi(d_2)$$
Cash-or-nothing put\\
$$P=e^{-rT}\Phi(-d_2)$$
Asset-or-nothing call\\
$$C=Se^{-qT}\Phi(d_1)$$
Asset-or-nothing put\\
$$P=Se^{-qT}\Phi(-d_1)$$
\end{frame}



\begin{frame}
\frametitle{Next Step}

\end{frame}
\end{document}
